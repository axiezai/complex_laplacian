\documentclass{article}


\usepackage{arxiv}

\usepackage[utf8]{inputenc} % allow utf-8 input
\usepackage[T1]{fontenc}    % use 8-bit T1 fonts
\usepackage{hyperref}       % hyperlinks
\usepackage{url}            % simple URL typesetting
\usepackage{booktabs}       % professional-quality tables
\usepackage{amsfonts}       % blackboard math symbols
\usepackage{nicefrac}       % compact symbols for 1/2, etc.
\usepackage{microtype}      % microtypography
\usepackage{lipsum}
\usepackage{graphicx}
\graphicspath{{./figures/}}
\usepackage{amssymb}
\usepackage{amsmath}
\usepackage{physics}
\usepackage{lineno}
\usepackage{adjustbox}
\usepackage{natbib}
\usepackage{bm}

\title{Emergence of Canonical Fuctional Networks from Complex Laplacian of Structural Connectome}


\author{
  Xihe Xie\thanks{Corresponding author} \\
  Department of Neuroscience\\
  Weill Cornell Medicine\\
  New York, NY 10028 \\
  \texttt{xix2007@med.cornell.edu} \\
  %% examples of more authors
   \And
  Chang Cai \\
  Department of Radiology and Biomedical Imaging\\
  University of California, San Francisco\\
  San Francisco, CA 94143\\
  \texttt{chang.cai@ucsf.edu} \\
   \And
  Pablo F. Damasceno \\
  Department of Radiology and Biomedical Imaging\\
  University of California, San Francisco\\
  San Francisco, CA 94143\\
  \texttt{pablo.damasceno@ucsf.edu}\\
  \And
  Srikantan Nagarajan \\
  %% Affiliation \\
  Department of Radiology and Biomedical Imaging\\
  University of California, San Francisco\\
  San Francisco, CA 94143\\
  \texttt{srikantan.nagarajan@ucsf.edu} \\
  \And
  Ashish Raj \\
  %% Affiliation \\
  Department of Radiology and Biomedical Imaging\\
  University of California, San Francisco\\
  San Francisco, CA 94143\\
  \texttt{ashish.raj@ucsf.edu} \\
  %% \And
  %% Coauthor \\
  %% Affiliation \\
  %% Address \\
  %% \texttt{email} \\
}

\begin{document}
\maketitle

\begin{abstract}
Human brain connectivity obtained via diffusion weighted imaging represent an approximate quantification of the brain's white-matter fiber tract network, however the mechanism that produces functional brain networks from underlying white-matter connections is still unclear. Efforts in correlational analysis, effective connectivity, and graphical models of brain networks have all provided insight into how the brain's function is linked to its structure. Here, we make use of both the structural connectome and its corresponding fiber tract distance adjacency matrix to summarize how resting-state functional activation patterns arise from the underlying structural connections of the brain. The distance adjacency matrix introduces a delay to signals being transmitted in the network, from which we are able to extract complex eigen basis of the graph Laplacian. In this work, we show that the structural eigenmodes of the brain's complex Laplacian matrix, without any functional neural activity modeling, is sufficient in reproducing the spatial patterns of canonical functional networks in the human brain. Additionally through optimizing the parameters controlling the complex structural eigenmodes, we show that the structural eigenmodes of the complex Laplacian outperforms the real-valued Laplacian eigenmodes, and particular canonical functional networks have an occupancy preference to specific subsets of the complex structural eigenmodes. 
\end{abstract}

% keywords can be removed
\keywords{structural connectivity \and functional networks \and graph Laplacian \and complex Laplacian}

\section{Introduction}
% modeling at different scales for SC-FC relationships
The exploration of structure and function relationships is a fundamental scientific inquiry at all levels of biological organization, and the structure-function relationship of the brain is of immense interest in neuroscience. Attempts at mathematical formulations of neuronal activity began with describing currents traveling through a neuron's membranes and being charged via ion channels\cite{hodgkin_quantitative_1952}. Recently, the focus of computational models have expanded from small populations of neurons to macroscale brain networks. Diffusion-weighted magnetic resonance imaging (dMRI) and functional magnetic resonance imaging (fMRI) data exhibits detailed whole brain white-matter tracts as the brain's structural connectivity (SC) and correlated activation patterns over time as functional connectivity (FC), respectively. Such high resolution images of the brain also allowed neuroscientists to label the brain according to anatomical or functional regions of interest (ROIs) \cite{Desikan2006, craddock_whole_2012}. Subsequently, efforts in graphical modeling have emerged as an effective theoretical tool to study the brain's SC-FC relationship based on the parcellated brains: ROIs became nodes and connectivity strength became edges on the graph, while dynamical systems describing neuronal activity played out on such graphs. 

%%% paragraph on what's been done in graphical modeling 
There have been a diverse range of graph based methods in relating the brain's structure to function. Particularly, perturbations and evolution of the structural and functional networks were investigated using both graph theoretical statistics \cite{Kuceyeski2016} as well as network controllability \cite{gu_controllability_2015, Muldoon2016}. And structurally informed models of brain function have been optimized to achieve high correlation between simulated and empirical functional patterns \cite{Honey2009}. Then dynamical causal modeling emerged as a powerful tool to infer effective (directional) connectivity with neurobiological models \cite{frassle_generative_2018}. Lastly, much simper graphical models utilizing the Laplacian spectra have revealed it to be an advantageous tool in relating SC to FC \cite{Abdelnour2018}, mainly due to low-dimensionality and more interpretable analytical solutions.

%%% paragraph on canonical functional networks + atasoy/ de ville efforts
The Laplacian matrix representation of a network can be used to find characteristic properties of the network \cite{Stewart1999}, and its eigenmodes (or spectral basis) are the orthonormal basis that represent particular transmission patterns on the network. Recent works have utilized the SC's Laplacian eigenmodes as the substrate on which functional patterns of the brain
is circulated \cite{Atasoy2016, preti_decoupling_2019}. These works mainly focused on replicating canonical functional networks, which are stable large scale circuits made up of functionally distinct ROIs distributed across the cortex that were extracted by clustering a large fMRI dataset \cite{Yeo2011}. In other words, the 7 canonical functional networks revealed are the best estimate of the functional organization of the human brain. If these FC patterns are most prominently identifiable networks in the human brain, will the same patterns emerge by only looking at the structural information of the brain?

%the functional organization of the brain can be mapped with such high confidence, then how well can SC Laplacian eigenmodes 

%%% this work is better cuz blah blah

%% end of intro

\section{Methods}
\label{sec:methods}

\paragraph{Notation.} In our notation, vectors and matrices are represented in \textbf{bold}, and scalars by normal font. We denote frequency of a signal, in Hertz (Hz), by symbol $f$, and the corresponding angular frequency as $\omega = 2 \pi f$. The structural connectivity matrix is denoted by $\bm{C} = c_{jk}$, consisting of connection strength $c_{jk}$ between any two pairs of brain regions $j,k$.

\subsection{Structural Connectivity Networks} We constructed structural connectivity networks according to the Desikan-Killiany atlas where the brain images were parcellated into 68 cortical regions and 18 subcortical regions as available in the FreeSurfer software \cite{Fischl2002, Desikan2006}. We first obtained openly available diffusion MRI data from the MGH-USC Human Connectome Project to create an average template connectome \cite{McNab2013}. Additionally, we obtained individual structural connectivity networks from 36 subjects' diffusion MRI data. Specifically, \textit{Bedpostx} was used to determine the orientation of brain fibers in conjunction with \textit{FLIRT}, as implemented in the \textit{FSL} software \cite{Jenkinson2012}. Tractography was performed using \textit{probtrackx2} to determine the elements of the adjacency matrix. We initiated 4000 streamlines from each seed voxel corresponding to a cortical or subcortical gray matter structure and tracked how many of these streamlines reached a target gray matter structure. The weighted connection betwen the two structures $c_{i,j}$ was defined as the number of streamlines initiated by voxels in regions $i$ that reach any voxel within region $j$, normalized by the sum of the source and target region volumes ($c_{i,j} = \frac{streamlines}{v_i + v_j}$). This normalization prevents large brain regions from having extremely high connectivity due to having initiated or received many streamline seeds. Afterwards, connection strengths are averaged between both directions ($c_{i,j}$ and $c_{j,i}$) to form undirected edges. Additionally, to determine the geographic location of an edge, the top 95\% of non-zero voxels by streamline count were computed for both edge directions, The consensus edge was defined as the union between both post-threshold sets. As a result, we obtain an undirected, weighted graph representation of the following matrix:

\[ \mathbf{C} = \begin{cases}
    0 & i = j\\
    c_{i,j} & otherwise
    \end{cases}
    \]

\subsection{Complex Graph Laplacian Eigenmodes}
For the undirected, weighted graph representation of the structural network $c_{i,j}$, we model the average neuronal activation signal for the $i$-th region as $x_{i}(t)$:

\begin{equation} \label{eq1}
\frac{dx_{i}(t)}{dt} = -\beta (x_{i}(t) - \frac{\alpha}{\sqrt{deg_i}\sqrt{deg_j}} \sum_{i,j} c_{i,j} x_{j}(t-\tau^{\nu}_{i,j}))
\end{equation} 

Where the first term is an exponential decay describing the refractory period after neuronal discharge with a rate constant $\beta$. The second term is the input signal from the $j$-th regions connected to region $i$ scaled by the connection strength from $c_{i,j}$, normalized by row and column degree, and delayed by $t-\tau^{\nu}_{i,j}$. The term $\tau^{\nu}_{i,j}$ is the delay in seconds obtained by dividing the distance adjacency matrix $D$ by transmission velocity $\nu$. The parameter $\alpha$ acts as both a global coupling parameter as well as a diffusion rate constant to distinguish the diffusion dynamics from $\beta$. The delays between connected brain regions turns into phase shifts in the frequency profiles of the oscillating signals, we transform \ref{eq1} into the frequency domain by utilizing the Fourier transforms $\frac{dx_{i}(t)}{dt} \to j\omega x_{i}(\omega)$ and $x(t-\tau^{\nu}_{i,j}) \to e^{-j\omega \tau^{\nu}_{i,j}} x_{i}(\omega)$:

\begin{equation}
\label{eq2}
j\omega x_{i}(\omega) = -\beta x_{i}(\omega) + \frac{\alpha}{\sqrt{deg_i}\sqrt{deg_j}} \sum_j c_{i,j} e^{-j\omega \tau^{\nu}_{i,j}} x_{i}(\omega)
\end{equation}

Where oscillatory frequency $\omega = 2 \pi f$. Instead of following individual regions $x_{i}$, we re-write (\ref{eq2}) in matrix form so that $\bar{X}(\omega)$ includes all brain regions as its rows and each frequency bin's magnitude as it's columns. Additionally, we combine the degree normalization term $\sqrt{deg_i}\sqrt{deg_j}$ as $\Delta$ and define a complex connectivity term $C^{*}(\omega) = \Delta^{-1} c_{i,j}e^{-j\omega \tau^{\nu}_{i,j}}$. Then (\ref{eq2}) becomes:

\begin{equation}
\label{eq3}
j\omega \pmb{\bar{X}}(\omega) = -\beta (\pmb{I} - \alpha \pmb{C}(\omega)) \pmb{\bar{X}}(\omega)
\end{equation}

In Eq.(2), the variables $\omega$ and $\tau_{i,j}^{\nu}$ are products in the same term, to avoid unidentifiable parameters, we combine these variables into one parameter named wave number $k$. Then by factoring out the diffusion constant $\beta$ from both terms, and having $\alpha$ act solely as the global coupling parameter, we obtain the expression $\mathcal{L} = (\omega) = \pmb{I} - \alpha \pmb{C}(\omega)$, where $\pmb{I}$ is the identity matrix and $\mathcal{L}(\omega)$ is the complex Laplacian matrix. Subsequently, the eigenvalues and eigenmodes of the complex Laplacian matrix can be obtained via the following decomposition:

\begin{equation}
\label{eq4}
\bm{\mathcal{L}}(\alpha, k) = \bm{U}(\alpha, k)\bm{\Lambda}(\alpha, k)\bm{U}(\alpha, k)^{H}
\end{equation}

Where $\bm{\Lambda}(\alpha, k) = diag([\lambda_{1}(\alpha, k), ... , \lambda_{N}(\alpha, k)])$ is a diagonal matrix consisting of eigenvalues of the complex Laplacian matrix and $\bm{U}(\alpha, k)$'s are the eigenmodes of the complex Laplacian matrix. The eigenmodes consist of values that represent the relative amount of activation in each parcellated brain region. These eigenmodes are tunable by the set of parameters $\alpha$ and $k$, which represent the global coupling strength between connections and wave number respectively. Wave number is defined as $k = \frac{\omega}{\nu}$, with $\nu$ being the transmission velocity for a signal in the network. The wave number is measured in cycles per unit distance, and can be interpreted as the spatial frequency of a wave \cite{French1971}. 

\subsection{Canonical Functional Networks}
We chose the canonical functional networks mapped by Yeo et al. \cite{Yeo2011} as the functional states most frequently visited by the human brain. The brain parcellations were created from fMRI recordings of 1000 young, healthy English speaking adults at rest with eyes open. A clustering algorithm was used to parcellate and identify consistently coupled voxels within the brain volume. The results revealed a coarse parcellation of seven networks: limbic, default, visual, frontoparietal, somatomotor, ventral attention, and dorsal attention. 

The canonical functional brain network parcellation was co-registered to brain regions of interest in the gyral based Desikan-Killany atlas \cite{Desikan2006} to match the dimensionality of our complex Laplacian structural eigenmodes. Then spatial activation maps of each canonical network was produced by normalizing the number of voxels per brain region belonging to a specific canonical network by the total number of voxels in the brain region of interest (Fig 1). Both the functional networks and the Desikan-Killiany atlas are openly available for download from Freesurfer \cite{Fischl2012} (http://surfer.nmr.mgh.harvard.edu/).

\subsection{Similarity Analysis}
With the HCP template connectivity matrix and it's corresponding distance adjacency matrix, we evaluated the spatial similarities between the seven canonical functional networks and the eigenmodes of the complex Laplacian. 

Firstly, we performed the "basin-hopping" global optimization technique \cite{Wales1997} to find a set of parameters that provided the highest spatial correlation value between each of the canonical functional network and the best performing complex Laplacian eigenmode. To ensure we achieve the global optimal in the nonlinear eigen decomposition process, we initiated the optimization procedure from ten different initial parameter values and selected the optimized parameters resulting from the optimization run that achieved the highest correlation value. While the basin-hopping algorithm always accepts new parameters that map cost function evaluations that improves upon the previous iteration, it will also accept solutions that does not improve upon the previous iteration to move out of local minima. The acceptance probability of this new iteration is $\exp(\frac{-(f(x)_{new} - f(x)_{old})}{T})$, where $f(x)_{new}$ and $f(x)_{old}$ are the current and previous cost function evaluations respectively. $T$ is the "temperature" parameter used in the acceptance/rejection criterion, larger $T$ values indicate that the algorithm is more willing to accept larger jumps in the cost function evaluation. We use a $T$ value of $0.01$ in our optimization schemes as that is a close estimate of the difference between our local minima.

By definition, the structural eigenmodes act as the basis activation patterns for a brain network, and we wanted to determine if a cumulative combination of these structural eigenmodes will improve the spatial similarity between a canonical functional network and a structural basis. Therefore, after acquiring a set of optimized parameters and structural eigenmodes that provided the highest spatial correlation values for each canonical functional network, we ranked all the structural eigenmodes from the most similar to least similar to each functional network's spatial patterns, and computed the weights that best solves the simple equation:
\begin{equation}
    \mathbf{Ax = B}
\end{equation}

The weights $\mathbf{x}$ is obtained by minimizing the 2-norm of $\mathbf{\norm{B-Ax}}$, where $\mathbf{A}$ is a matrix of structural eigenmodes and $\mathbf{B}$ is the vector representing spatial activation patterns of a particular canonical functional network. Spatial similarity was computed for all possible combination of eigenmodes. While Spearman's correlation was appropriate for non-continuous correlative comparisons, it's non-linearity due to sorting of values was evident, and Pearson's correlation provided much steadier results. Therefore, Spearman's correlation was used and reported for optimizing the best performing eigenmode, and both Spearman's and Pearson's were reported for cumulative comparisons.
% we cna also point out Yeo et al used Pearson's correlation of voxel-wise activation time series from fMRI as minimization metric in his study as well. 

Using the same set of optimized parameters when appropriate, we repeated this analysis for both a real-valued Laplacian without frequency and transmission speed tuning, as well as complex Laplacians obtained from 1000 null connectivity matrices. The null connectivity matrices are constructed with the same sparsity as the HCP template connectivity matrix, and the elements of the adjacency matrix are assigned by randomly sampling from a white-matter fiber tract distance based distribution calculated by $\exp{-\mathbf{distances}}$. The 1000 randomly sampled null connectivity matrices allow us to generate a null distribution of spatial correlation values and compare the performance of a brain graph's complex and real-valued Laplacian eigenmodes to random graph Laplacian eigenmodes.

\section{Result}

\subsection{Structural connectivity based functional activation patterns}
First, we use the HCP template connectome to demonstrate the complex Laplacian's additional transmission speed and oscillatory frequency parameters provide a wide range of spatial activation patterns. The top row of Figure 2 shows three exemplary structural eigenmodes with high coupling strength ($\alpha = 2.5, k = 2$), each eigenmode showing distinct spatial activation patterns: eigenmode 1 has high activity in the lateral temporal brain regions, whereas the medial brain regions of eigenmode 2 has the higher activations, and eigenmode 3 engages inferior brain regions along both the midline and lateral periphery of the brain. On the other hand, the mid row of Figure 2 shows exemplary structural eigenmodes with low coupling strength ($\alpha = 0.2,k = 8$), with eigenmodes 1 and 3 showing left and right hemisphere specific activation around the dorsal brain regions while eigenmode 2 recruits both the frontal and occipital regions in both hemispheres. Consistent with previous works of brain structural eigenmodes \cite{Atasoy2016}, we show that a real valued Laplacian without frequency and transmission speed tuning is capable of producing similar spatial patterns in the last row of Figure 2. The coupling strength parameter $\alpha$ is the only parameter in this case, and we find this overall scaling has very little effect on the elements of individual eigenmodes. All spheres in all glass-brain illustrations are normalized between 0 and 1. 

\subsection{Eigenmodes of the complex Laplacian exhibit canonical functional network activation patterns}
We re-assigned the voxel-wise parcellations of the seven canonical functional networks from Yeo et al. \cite{Yeo2011} to brain regions from the Desikan-Killiany atlas (Figure 3), this re-sampling of the parcellations allow spatial pattern comparisons of equal dimensions against our eigenmodes. The middle column of Figure 3 shows best matching eigenmodes after global optimization of HCP template connectome's eigenmodes to each of the functional networks. In addition to displaying the best performing eigenmode in each case, we also ranked the eigenmodes according to their spatial correlation values and displayed the linear combination of the top 10 performing eigenmodes on the right column of Figure 3. The spatial correlation values of the best performing eigenmode as well as cumulative combinations of eigenmodes are reported in Figure 5.

Specifically, canonical functional network patterns emerges when parameters are optimized for each network, producing a complete set of 86 structural eigenmodes. While we only tune the parameters for a single best performing structural eigenmode, which is shown in the middle column of Figure 3, combining additional eigenmodes from the same parameter set improves spatial similarity despite those subsequent eigenmodes being untuned to a particular canonical functional network. We reported spatial similarities to up to 30 combined eigenmodes in Figure 5, but linearly combining more eigenmodes up until we reach the complete basis of 86 eigenmodes steadily improves the Pearson's correlation value between each canonical functional network and the combined eigenmode spatial pattern.

% Possible discussion point % 
% and After ranking the structural eigenmodes by their spatial similarities to canonical functional networks, we find that the best performing eigenmodes act as dominant structural basis for each canonical functional network. Furthermore, combinations of additional high ranking eigenmodes improve spatial similarity. 

\subsection{Parameter tuning of complex Laplacian eigenmodes}
To examine the sensitivity of our eigenmodes to our complex Laplacian parameter, we first computed spatial correlation values for the best performing eigenmodes for each canonical functional network across the entire parameter range. From Figure 4, we see that the best achievable spatial correlation for all networks stay consistent as we change the global coupling strength parameter, whereas wave number tuning causes shifts in spatial similarity between the best performing eigenmode and canonical functional networks as well as which particular eigenmode achieves the highest spatial correlation. It is evident that we need to tune both the global coupling strength and wave number for a dominant eigenmode matching a specific canonical functional network to emerge. In both scenarios, the limbic network has the worst spatial match and the least amount of shift in spatial correlation values. 

To further examine the tunable parameter's effects on the dominant eigenmode, we show a heatmap of the spatial correlation achieved by the dominant eigenmode as we shifted parameter values in Supplementary Figure 1. As expected, global coupling parameter had no effect on dominant eigenmode's fit while the wave number did. Subsequently, we split the wave number parameter into its two components: signal transmission velocity and oscillatory frequency, showing that those two components equally affect spatial patterns emerging from the complex Laplacian eigenmodes (Supp. Figure 1 bottom row). The spatial correlation patterns of each functional network also implies that there are potentially functional network specific eigenmodes obtainable from the structural complex Laplacian, which will be explored further in the subsequent group level analysis.

\subsection{Complex Laplacian eigenmodes outperform real Laplacian eigenmodes}
To compare the performance of the complex Laplacian structural eigenmodes, we performed the same similarity analysis with the HCP connectome's real valued Laplacian eigenmodes as well as complex Laplacian eigenmodes of randomly sampled networks. We specifically created 1000 random networks and performed Fisher's r to z transform to produce the 95\% confidence intervals shown in Figure 6. For each combination of eigenmodes, we created a null distribution from the Fisher's r to z values and obtained Bonferroni corrected one-tail p-values for all null distribution comparisons. 

Overall, the complex Laplacian's best performing eigenmodes outperform real valued Laplacian eigenmodes in 6 out of the 7 canonical functional networks as shown in Figure 6. In comparison to the null distribution of randomly generated Laplacian eigenmodes, the complex Laplacian's top performing eigenmode significantly outperforms the random eigenmodes in all cases. The ventral attention network comparison has the highest p-value of $p=0.0044$, while all other comparisons achieved $p< 0.0001$. On the other hand, the real-valued regular Laplacian's top performing eigenmode significantly outperformed the random Laplacian eigenmodes when comparing the limbic, default mode, and visual networks only ($p<0.001$). When comparing the performances of the combination of 10 eigenmodes as shown in the right column of Fig 3, the complex Laplacian achieves $p<0.0001$ for all canonical functional network comparisons, while the regular Laplacian's comparisons yielded $p<0.0005$ for all networks except the ventral attention network ($p<0.01$).

For the cumulative combination of eigenmodes up to a total of 30 eigenmodes as shown in Figure 6, we see an overall upward trend of spatial correlation values with the Pearson's correlation metric. The Spearman's correlation values are extremely volatile in comparison, but shows overall increases in spatial correlation given enough combinations of eigenmodes. This volatile behavior is due to the non-linear sorting operations used when computing Spearman's correlation. Lastly, the spatial correlation values will approach close to a perfect match if given the complete set of eigenmodes (not shown).

%% add in discussion? - In other words, is a functioning brain able to recruit and tune neuronal activity as organized by basis structural patterns to access functional networks.
\subsection{Group level eigenmode analysis}
Figure 6 shows a violin plot of the best spatial correlation values achieved by each subject's complex valued Laplacian in orange and real valued Laplacian in blue. Paired T-tests were performed for all canonical functional networks, the complex Laplacian eigenmodes outperform the real value Laplacian eigenmodes on the group level for all networks except the limbic network. The paired T-tests between the two Laplacian types for all remaining networks had Bonferroni corrected p-values below $p = 0.01$. 

Lastly, with the exception of the default mode network, whose best performing structural eigenmodes span across the range of all eigenmodes numbers, all other canonical functional networks exhibit selectivity towards a specific subset of eigenmodes (Figure 7). The limbic and visual networks prefer to occupy structural eigenmodes at both low and high ends of the eigen spectrum, while the dorsal and ventral attention networks mainly occupy the middle of the eigen spectrum. The specific occupancy patterns shown in Figure 8 implies there may be a hierarchy to the canonical functional networks that are accessible by the underlying structural connections of the brain, and the functioning brain minimizes recruitment of unrelated structural connections when engaged in specific tasks. 

\section{Discussion}

\bibliographystyle{unsrt}  
\bibliography{references.bib}  

\end{document}
