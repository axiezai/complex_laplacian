\documentclass{article}


\usepackage{arxiv}

\usepackage[utf8]{inputenc} % allow utf-8 input
\usepackage[T1]{fontenc}    % use 8-bit T1 fonts
\usepackage{hyperref}       % hyperlinks
\usepackage{url}            % simple URL typesetting
\usepackage{booktabs}       % professional-quality tables
\usepackage{amsfonts}       % blackboard math symbols
\usepackage{nicefrac}       % compact symbols for 1/2, etc.
\usepackage{microtype}      % microtypography
\usepackage{lipsum}
\usepackage{graphicx}
\graphicspath{{./figures/}}
\usepackage{amssymb}
\usepackage{amsmath}
\usepackage{lineno}
\usepackage{adjustbox}
\usepackage{natbib}
\usepackage{bm}

\begin{document}
\section{Compiled Figures}

\begin{figure}[ht]
  \centering
  \includegraphics[width=0.9\linewidth]{../figures/fig1/fig1_overview_v3.pdf}
  \caption{The analysis overview. Structural connectivity matrix ($C$) and distance adjacency matrix ($D$) were extracted from diffusion MRI data, the values in the matrices were viewed as nodes and edges metrics in a network. We perform an eigen decomposition on the network's complex Laplacian ($\mathcal{L}$) to obtain structural eigenmodes of the brain ($U$), then we compare the spatial similarities between the structural eigenmodes and canonical functional networks. Here we show brain rendering of the first eigenmode from the HCP template connectivity matrix (yellow spheres) and the visual functional network (blue spheres) as an example. The size of the spheres represent the amount of participation in a particular network from each brain region.}
  \label{fig:fig1}
\end{figure}

\begin{figure}[ht]
 \centering
 \includegraphics[width = \linewidth]{../figures/fig2/eigmode_examples.pdf}
 \caption{Laplacian eigenmode examples. Here we illustrate three representative eigenmodes from the complex Laplacian with different tuning parameters as well as three representative eigenmodes from the real Laplacian without transmission speed and distance delay properties. The top and middle row shows brain renderings of the complex Laplacian eigenmodes with {$\alpha = 2.5, k = 2$} and {$alpha = 0.2, k = 8$} respectively. The real Laplacian eigenmodes in the bottom row does not have tunable parameters, as coupling strength tuning has minimal effect on the spatial patterns of the eigenmodes after degree normalization. The size and color of the spheres represent the relative amount of participation in the network for each brain region.}
 \label{fig:fig2}
\end{figure}

\begin{figure}[ht]
 \centering
 \includegraphics[width = \linewidth]{../figures/fig3/fig3_compiled.pdf}
 \caption{Canonical functional networks reproduced by structural eigenmode. Brain renderings of the seven canonical functional networks are listed in the left column. Parameter optimization identified individual structural eigenmodes with the highest spatial correlation to each functional network as shown in the middle column. After ranking all structural eigenmodes by highest spatial correlation, a linear combination of the top ten best performing eigenmodes are shown in the right column.}
 \label{fig:fig3}
\end{figure}

\begin{figure}[ht]
\centering
\includegraphics[width = 0.9\linewidth]{../figures/fig4/parameters.pdf}
\caption{Structural eigenmode spatial similarity to canonical functional networks has parameter dependency. Heatmaps displaying spatial correlation values of all complex Laplacian eigenmodes span by parameters values for each network. Coupling strength ($\alpha$, top) and wave number ($k$, bottom) all exhibit control over structural eigenmode's occupation of certain functional networks. Specifically, shifts in parameter values can cause a change in selection of the best matching structural eigenmode.}
\label{fig:fig4}
\end{figure}

%%%%%%%%%%%%%%%%%%%%%%%%%%%%%%%%%%%%%%%%%%%%%%%%%%%%%%%%%
%\caption{Parameter control of best performing eigenmodes. The highest spatial correlation values achieved by the best performing eigenmode for each functional network is shown as a function of tuning parameters. Coupling strength ($\alpha = [0.1, 5]$) does not affect the eigenmodes' spatial patterns while oscillatory frequency ($f = [2, 45]$ Hz) does as shown on top. On the other hand, The spatial patterns of the structural eigenmodes are more sensitive to transmission velocity ($\nu = [5, 20]$ m/s) as shown on the bottom row. Colorbar indicates spatial similarity as measured by Spearman's correlation.}
%%%%%%%%%%%%%%%%%%%%%%%%%%%%%%%%%%%%%%%%%%%%%%%%%%%%%%%%%

%%%%%%%%%%%%%%%%%%%%%%%%%%%%%%%%%%%%%%%%%%%%%%%%%%%%%%%%%
%%%%%%%%%% This figure is moved to Supplements %%%%%%%%%%
%%%%%%%%%%%%%%%%%%%%%%%%%%%%%%%%%%%%%%%%%%%%%%%%%%%%%%%%%
%\begin{figure}[ht]
%\centering
%\includegraphics[width = 0.9\linewidth]{../figures/fig5/f5_compiled.pdf}
%\caption{Parameter control of best performing eigenmodes. The highest spatial correlation values achieved by the best performing eigenmode for each functional network is shown as a function of tuning parameters. Coupling strength ($\alpha = [0.5, 4.5]$) does not affect the eigenmodes' spatial patterns while wave number ($k = [0, 100]$) does as shown on top. Additionally, the spatial patterns of the structural eigenmodes are sensitive to both factors contributing to wave number (transmission velocity ($\nu = [5, 20]$ m/s) and oscillatory frequency ($f = [2, 47]Hz$)) as shown on the bottom row. Colorbar indicates spatial similarity as measured by Spearman's correlation.}
%\label{fig:fig5}
%\end{figure}
%%%%%%%%%%%%%%%%%%%%%%%%%%%%%%%%%%%%%%%%%%%%%%%%%%%%%%%%%%

\begin{figure}[ht]
\centering
\includegraphics[width = 0.9\linewidth]{../figures/fig5/spatial_correlation.pdf}
\caption{Structural eigenmodes of the HCP template complex Laplacian predicts canonical functional networks better than structural eigenmodes of the real Laplacian. For each canonical functional network, we compared its spatial patterns with eigenmodes obtained from the complex Laplacian (orange), real Laplacian (blue), complex Laplacians obtained from 1000 permutations of random connectivity matrices (green). The comparisons were performed with both Spearman's correlation (top) and Pearson's correlation (bottom). In all cases, the spatial similarity between canonical functional networks and structural eigenmodes increase as we cumulatively combine structural eigenmodes. The complex Laplacian eigenmodes were able to outperform the random Laplacian in all cases with just the best performing eigenmode. However, the real Laplacian required combinations of eigenmodes to outperform the random Laplacian in some cases. Lastly, eigenmodes of the complex Laplacian are more similar to canonical functional networks than eigenmodes of the real Laplacian with the Limbic network being the only exception.}
\label{fig:fig5}
\end{figure}

\begin{figure}[ht]
\centering
\includegraphics[width = 0.9\linewidth]{../figures/fig6/reg_com_laplacian_subjects_correlations.pdf}
\caption{Complex Laplacian outperforms real Laplacian in recapitulating canonical functional networks. Violin plot showing that on a group level (each dot correspond to one subject, $n = 36$), the best performing structural eigenmodes of the complex Laplacian outperforms the corresponding structural eigenmode from the real Laplacian. The only group comparison that failed to achieve significance under a paired t-test is the limbic network ($p = 0.64$).}
\label{fig:fig6}
\end{figure}


\begin{figure}[ht]
\includegraphics[width = 0.9\linewidth]{../figures/fig7/eigen_numbers.pdf}
\caption{Canonical functional networks have complex Laplacian eigenmode specificity. Each dot on the violin plot corresponds to the best performing eigenmode number. Showing that across all subjects ($n = 36$), canonical functional networks occupies specific structural eigenmodes as the dominant structural basis. Default mode network is the exception as the best performing eigenmode spans across all eigenmodes. On the other hand, the rest of the canonical functional networks cluster to specific eigenmode numbers.}
\label{fig:fig7}
\end{figure}
\end{document}
